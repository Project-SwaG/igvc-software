\section{Mechanical Design}

\subsection{Structure \& Vehicle Layout Overview}
% Talk about key constraints drving structure and layout. Front zone, middle zone, and rear zone.

\begin{figure}[H]
\begin{center}
\includegraphics[width=3in]{./pics/RobotFrontCover.png}
\caption{Roxii, the 2011-2012 Base}
\label{FIG:Roxii}
\end{center}
\end{figure}


  During system debrief and review, several actionable deficiencies and areas of improvement for last years mechanical platform were noted. These included poor electronics serviceability and weather resistance issues. This years' drive base design, shown in Figure \ref{FIG:Roxii} was guided by the following main goals:

\begin{enumerate}
\item Increased adverse weather performance
\item Outer panel simplification
\item Improved electronics access
\item New sensor mounts for GPS and Magnetometer
\end{enumerate}

To this end a the mechanical platform has been developed which features a more robust exterior paneling system, a lighter weight construction, improved electronics accommodation, new sensor mounts for GPS and magnetometer, and a low cost suspension. Overall our unit is 32 inches wide and 40 inches long. A three zone layout is uesd with the structure being composed of a front, middle, and rear zone. As illustrated in Figure \ref{FIG:Zones}, their content is as follows:

\begin{figure}[H]
\begin{center}
\includegraphics[width=3in]{./pics/RobotZones.png}
\caption{Robot Zones}
\label{FIG:Zones}
\end{center}
\end{figure}

\begin{enumerate}
\item \textcolor{blue}{Front}: Forward LIDAR, Motors, \& Power support for laptop
\item \textcolor{Orange}{Middle}: Main Batteries, Motor Drive Electronics, \& Power Distribution
\item \textcolor{green}{Rear}: Laptop, Camera, Rear LIDAR, Motors, GPS, Magnetometer, Button Panel, \& Safety Light
\end{enumerate}

The frame is constructed out of 1/16" wall 1" sq. steel tube and aluminum panels. Much of the assembly was accomplished with MIG welding and the use of 1/4-20 fasteners. The outer cover is again made from polycarbonate panels which have a new attachment method that aides in rain-proofing. Overall the system has retained many of the improvements made last year while adding key enhancements which are presented here in. 

\subsection{Drive System}
\subsubsection{Motor Configuration \& Suspension}
% We have 4 wheel independent suspension. Talk abut custom modification to off the shelf dampers.
As with last year our platform features an independently suspended four wheel drive system. In designing the shock absorber system, our team consulted with the FormulaSAE and BajaSAE teams which we share our shop with to gain a better understanding of the shocks they utilize and what characteristics we were looking for in our system. With cost in mind we purchased some dampers for riding lawn mower seats and modified them by adding springs to their stroke. These custom shocks have become a simple yet highly costs effective solution for our needs. A close up of one of these shocks can be seen on in the CAD rendering and picture in Figure \ref{FIG:Shock}.

\begin{figure}[H]
\begin{center}
\includegraphics[width=3in]{./pics/shock.png}
\caption{Shock System}
\label{FIG:Shock}
\end{center}
\end{figure}

\subsubsection{Motor \& Encoder Modification}
% Motor specs and part number of encoders. How we mounted encoders.
As with last year, this years base is powered by four NPC T64 brushed motors. Each motor has a custom adapter plate and shaft mounted on the rear for attaching a US Digital encoder. These encoders allow for independent control of each motor by the electronics system presented later in this paper. Figure \ref{FIG:motorencoder} shows the rear of a modified motor with the attached encoder.

\begin{figure}[H]
\begin{center}
\includegraphics[width=3in]{./pics/encodermotor.png}
\caption{Motor \& Encoder Closeup}
\label{FIG:motorencoder}
\end{center}
\end{figure}

In all the T64 motors coupled with foam filled wheels allow our platform to be more than capable of handling ramps and muddy segements of the couse often encountered after rain storms during competition. 

\subsection{Rain Proofing}
% Discuss waterproofing method with outter panel reduced complexity.
Rain proofing ensures the robot is able to operate consistantly in adverse weather conditions without damage to the electronic components. All removeable access panels are joined into a single piece by waterproof full length hinges, preventing leaking where panels join together. Around the frame at each access panel is a full length compressable foam gasget that prevent seeping if any water happens to pool on the top of the robot. Waterproof plugs and glands were also introduced to minimize water ingress at cable entry points.

\subsection{Electronics Accommodation}
\subsubsection{Electronics stack}
% Talk about new laptop size and how it fits. Expasion on previous design. Ease of access, etc.
One of the biggest improvement to our system this year is new organizational hardware for the internal electronics. Last year the electronics stacks were difficult to remove due to poor wire routing. This year, the electronics stack are easily removable from the robot interior. The batteries were lifted, allowing space for wires to travel underneath. Additionally, better optimization of space near the suspension mounts allowed for more space for wires. This additional space along with higher strand count wire greatly improved the mobility of the stacks. Alignment pins hold the stacks in place during movement, but allow the stacks to be lifted when changes to the electronics are necessary, and UHMW stands slide out from the frame to hold the stacks in a upright position for servicing.

\subsubsection{LIDAR}
% Mention LIDAR FOV protection from rain, accessibility. How it comes off with 3 bolts, etc.
The modified SICK NAV 200 LIDARs are mounted in the front and rear of the vehicle and each have an unobstructed sweep of roughly 255 degrees. The LIDARs are protected above from rain by overhangs which also serve as compartments for the laptop and electrical power system. The units are attached to the base through an intermediate aluminum plate. This allows them to be removed from the vehicle with only loosening three 1/4-20 bolts which are the standard fastener for our platform.
